\section{Introduction}
Relational Databases and NoSQL are two competing types of data models, and since the 1970's, the relational database model were the industries dominating storage method, up until the introduction of NoSQL in the early 2000, were NoSQL began replacing it in some applicational contexts \parencite{a-brief-history-on-non-relational-databases}.   
The evolution of new technologies and the era of Web 2.0, changed the way we interact online, which triggered the need for a more versatile and scalable database model \parencite{a-brief-history-of-nosql}.  
The introduction of NoSQL introduced a new way of dynamically storing data in a database, which led to the development of applications such as social media, e-commerce platforms, messaging software etc. where large amount of data are processed every second.

The shift from static to dynamic, caused many companies to switch part of their applications needs from the oled conventional method of using relational databases, to the more easily scalable and versatile type of database, like NoSQL.
However, in spite of NoSQL being the new "trend" and applications becoming more and more dynamic, the relational database model are still widely used today, but why is that?

This paper is based on reviews and research of past literature and begins with a description of both relational and non-relational databases, including examples of use-cases for each model. The discussion then progresses to comparing the features between the two models and then concludes with a direct comparison, where we, based on our findings, attempt to determine wether NoSQL is a direct replacement for the relational-database model, and which one you should choose for your next project.  