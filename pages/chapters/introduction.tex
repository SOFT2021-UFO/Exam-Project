\section{Introduction}
Relational Databases and NoSQL are two competing types of data models, and since the 1970's, the relational database model were the industries dominating storage method, up until the introduction of NoSQL in the early 2000, were NoSQL began replacing it in some applicational contexts \parencite{a-brief-history-on-non-relational-databases}.   
The evolution of new technologies and the era of Web 2.0, changed the way we interact online, which triggered the need for a more versatile and scalable database model\parencite{a-brief-history-of-nosql}.  
The introduction of NoSQL introduced a new way of dynamically storing data in a database, which led to the development of applications such as social media, e-commerce platforms, messaging software etc. where large amount of data are processed every second.

The shift from static to dynamic, caused many companies to switch part of their applications needs from the old conventional method of using relational databases, to the more easily scalable and versatile type of database, like NoSQL.
However, in spite of NoSQL being the new "trend" and applications becoming more and more dynamic, the relational database model are still widely used today, but why is that?

The intent of this paper is to describe what the two different database structures are and weigh them up against each other to form a baseline, for which can then be used when deciding on which one that suits future project the best.